\subsection{Instruction: skip}

\subsubsection{Summary}

\InputIfFileExists{instr_skip_summary.tex}{}{}

\subsubsection{Grammar and Compiling}

\InputIfFileExists{instr_skip_compiling.tex}{}{}

\subsubsection{Assembly Syntax}

\begin{myquote}
\begin{verbatim}
SKIPINSTR <- 'skip' S UNSIGNED
S <- %s+
UNSIGNED <- [0-9]+
\end{verbatim}
\end{myquote}

\InputIfFileExists{instr_skip_assembly.tex}{}{}

\subsubsection{Bytecode Encoding}

This instruction has a size of 8 bytes and is structured in bytecode as follows:

%DEADBEEF
$_{00}$\ 
\fbox{%
  \parbox{20pt}{%
00
  }%
}
\fbox{%
  \parbox{20pt}{%
04
  }%
}
\fbox{%
  \parbox{20pt}{%
03
  }%
}
\fbox{%
  \parbox{20pt}{%
30
  }%
}



$_{04}$\ 
\fbox{%
  \parbox{20pt}{%
00
  }%
}
\fbox{%
  \parbox{20pt}{%
00
  }%
}
\fbox{%
  \parbox{20pt}{%
00
  }%
}
\fbox{%
  \parbox{20pt}{%
00
  }%
}

%DEADBEEF
\InputIfFileExists{instr_skip_bytecode.tex}{}{}

\subsubsection{Execution State Change}

.

\InputIfFileExists{instr_skip_state.tex}{}{}

