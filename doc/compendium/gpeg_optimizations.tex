\section{Optimizations}

A PEG compiler only needs to be able to emit the following instructions to be
functionally complete:

\begin{itemize}
\item any
\item backcommit
\item call
\item char
\item closecapture
\item commit
\item end
\item fail
\item failtwice
\item intrpcapture
\item maskedchar
\item opencapture
\item partialcommit
\item ret
\item set
\item var
\end{itemize}

And even about this set, a discussion is possible:
Because theoretically, the 'char' and 'any' instructions are implied by
the 'set' instruction
(although that would bring with it a lot of bytecode overhead).
Also, 'intrpcapture', 'maskedchar' are specific to gpeg and
binary parsing.

Nevertheless, it's sometimes easy and efficient (both from a bytecode
size and execution speed perspective) to optimize the code somewhat.
Which is why the gpeg instruction set is somewhat bigger than the
list above.

\subsection{Optimizing Loops}
A matcher that is quantified for a range can be written out in
full, but gpeg introduces a set of counter registers for this operation,
as well as two instructions to use them: 'counter' and 'condjump'.
Grammar:

\begin{myquote}
\begin{verbatim}
'a'^3
\end{verbatim}
\end{myquote}

Is formally compiled to:

\begin{myquote}
\begin{verbatim}
  char 61
  char 61
  char 61
\end{verbatim}
\end{myquote}

And will be optimized in gpeg as:

\begin{myquote}
\begin{verbatim}
  counter 0 3
looplabel:
  char 61
  condjump 0 looplabel
\end{verbatim}
\end{myquote}

This optimization starts to make a lot more sense when the matcher
is quantified with values much higher than three. Which is why
gpeg performs this optimization automatically.

\subsection{Optimizing Sequences of Char}

gpeg has an instruction to match four bytes at once, called 'quad'.
To optimize matching larger string literals, these are chopped up into
chunks of four bytes, which each emit a 'quad' instruction,
and the remainder is emitted as char instructions.
Grammar:

\begin{myquote}
\begin{verbatim}
'gpeg'
\end{verbatim}
\end{myquote}

Is compiled to:

\begin{myquote}
\begin{verbatim}
  char 4e
  char 61
  char 69
  char 67
  char 61
  char 6d
  char 61
\end{verbatim}
\end{myquote}

May be optimized as:

\begin{myquote}
\begin{verbatim}
  quad 4e616967
  char 61
  char 6d
  char 61
\end{verbatim}
\end{myquote}

\subsection{Optimizing Sets}
When a set consists of a single range, it can be rewritten as a 'range'
instruction.

\begin{myquote}
\begin{verbatim}
[a-z]
\end{verbatim}
\end{myquote}

Compiles to:

\begin{myquote}
\begin{verbatim}
  set 000000000000000000000000feffff0700000000000000000000000000000000
\end{verbatim}
\end{myquote}

May be optimized as:

\begin{myquote}
\begin{verbatim}
  range 97 122
\end{verbatim}
\end{myquote}

\subsection{Optimizing Unlimited Sets}

Span

\subsection{Optimizing Dot-Quantified}

Skip

\subsection{Optimizing Tests}
