The 'catch' instruction is emitted when a fail condition has to be
caught. Currently, in gpeg, when:

\begin{itemize}
\item a list of matcher alternatives is defined to produce (part of) a rule
      (through the '/' operator). See example 1 below.
      Here, the failure of RULE1 is caught by subsequently trying RULE2.
      Only when RULE2 also fails, the production of EXAMPL1 fails.
\item a scan operation (through the '\&' and '!' operators) is performed.
      See example 2 below. RULE1 is evaluated, but if it fails, it succeeds,
      therefore its failure must be caught.
\item inconsequential ranges of matchers (through quantifiers) must be skipped.
      See example 3 below. All the failures to match RULE1 through times
      4-8 are not consequential to the success of the rule (although they
      do advance the input pointer), therefore any fail condition must be
      caught for these cases.
\end{itemize}

\begin{myquote}
\begin{verbatim}
EXAMPLE1 <- RULE1 / RULE2
EXAMPLE2 <- ! RULE1
EXAMPLE3 <- RULE1^3-8
\end{verbatim}
\end{myquote}

Assembly Patterns

\begin{myquote}
\textit{-- EXAMPLE1} \newline
\textit{\textbf{catch ALT} -- On failure, jump to calling RULE2} \newline
\textit{call RULE1} \newline
\textit{commit OUT -- Success, remove catch element and jump to OUT} \newline
\textit{\textbf{ALT}:} \newline
\textit{call RULE2} \newline
\textit{commit \_\_NEXT\_\_} \newline
\textit{OUT:} \newline
\end{myquote}

\begin{myquote}
\textit{-- EXAMPLE2} \newline
\textit{\textbf{catch SUCCESS} -- ! means failure is success} \newline
\textit{call RULE1} \newline
\textit{failtwice -- ! means success is failure} \newline
\textit{\textbf{SUCCESS}:} \newline
\end{myquote}

\begin{myquote}
\textit{-- EXAMPLE3} \newline
\textit{counter 0 3} \newline
\textit{LOOP1:} \newline
\textit{call RULE1} \newline
\textit{condjump 0 LOOP1} \newline
\textit{counter 0 5 -- 8 minus 3 is 5} \newline
\textit{LOOP2} \newline
\textit{\textbf{catch OUT}} \newline
\textit{call RULE1} \newline
\textit{partialcommit \_\_NEXT\_\_} \newline
\textit{condjump 0 LOOP2} \newline
\textit{commit \_\_NEXT\_\_} \newline
\textit{\textbf{OUT}:} \newline
\end{myquote}
