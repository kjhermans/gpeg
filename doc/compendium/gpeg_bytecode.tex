\section{Bytecode}
\label{sec:bytecode}

\subsection{Instruction sets}

gpeg, generation 3, provides two instruction sets (with a few overlapping
instructions) that each have their own stack and other memories:
one for matching, and one for script execution.


%\begin{table}[]
\begin{center}
%\caption{gpeg Bytecode Instructions}
\label{tab:naig_bytecode}
\begin{longtable}{lllll}
\textbf{Mnemonic} & \textbf{Opcode} & \textbf{Param1} & \textbf{Param2} & \textbf{Length} \\
\endhead
any & 000003e4 &  &   & 4 \\
backcommit & 000403c0 & address &   & 8 \\
bitmask & 00100365 & nbits & bits  & 20 \\
call & 00040382 & address &   & 8 \\
catch & 00040393 & address &   & 8 \\
char & 000403d7 & char &   & 8 \\
closecapture & 00040300 & slot &   & 8 \\
commit & 00040336 & address &   & 8 \\
condjump & 00080321 & register & address  & 12 \\
counter & 00080356 & register & value  & 12 \\
end & 000400d8 & code &   & 8 \\
fail & 0000034b &  &   & 4 \\
failtwice & 00000390 &  &   & 4 \\
intrpcapture & 0008000f &  &   & 12 \\
jump & 00040333 & address &   & 8 \\
noop & 00000000 &  &   & 4 \\
opencapture & 0004039c & slot &   & 8 \\
partialcommit & 000403b4 & address &   & 8 \\
quad & 0004037e & quad &   & 8 \\
range & 000803bd & from & until  & 12 \\
ret & 000003a0 &  &   & 4 \\
set & 002003ca & set &   & 36 \\
skip & 00040330 & number &   & 8 \\
span & 002003e1 & set &   & 36 \\
testany & 00040306 & address &   & 8 \\
testchar & 0008039a & address & char  & 12 \\
testquad & 000803db & address & quad  & 12 \\
testset & 00240363 & address & set  & 40 \\
trap & ff00ffff &  &   & 4 \\
var & 000403ee & slot &   & 8 \\
\end{longtable}
\end{center}
%\end{table}


\subsection{Bytecode Structure}

A gpeg bytecode file or buffer consists of a sequence of binary
instructions which, in turn, each consist of a binary
encoded opcode, plus their parameters, should they have any.

The amount and kind of parameters following an opcode, is strictly
defined:
the same opcode will always be followed by the same kinds of parameters
and therefore, an instruction type will always be the same size
(see table [\ref{tab:naig_bytecode}]).

gpeg bytecode is taken by the gpeg engine program (gpege) or
library, and run against an input, to produce an output.

\subsection{Opcode Values}

Opcode values are determined through

\begin{itemize}
\item Grouping; 
\item Hamming distance;
\item Instruction size;
\end{itemize}

\subsection{Noop Slides and Canaries}

Implementations that want each instruction to have exactly the
same size, can choose to pad the encoding of shorter instructions
with either no-ops, or canaries.

\subsection{Encoding of Parameters}

\subsubsection{Address}

\subsubsection{Char}

\subsubsection{Slot}

\subsubsection{Register}
