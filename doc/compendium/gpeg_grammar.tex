\section{Grammar}
\label{sec:grammar}

A gpeg grammar file or buffer consists either of a single
unnamed expression, or of a sequence of named expressions (called rules),
whitespace and comments, denoted in ASCII text.
gpeg grammar is defined with the express purpose of matching, validating
and/or capturing from an arbitrary data input.
gpeg grammar is taken by the gpeg compiler program (naic) and turned
into gpeg assembly [section \ref{sec:assembly}].
In this, it resembles greatly other grammar description and parsing methods
such as Backus-Naur, Regular Expressions, Lex and Yacc.

\begin{myquote}
\begin{verbatim}
-- JSON; JavaScript Object Notation.
TOP          <- JSON
__prefix     <- %s*
JSON         <- HASH END
HASH         <- CBOPEN OPTHASHELTS CBCLOSE
OPTHASHELTS  <- HASHELTS / ...
HASHELTS     <- HASHELT COMMA HASHELTS / HASHELT
HASHELT      <- STRING COLON VALUE
ARRAY        <- ABOPEN OPTARRAYELTS ABCLOSE
OPTARRAYELTS <- ARRAYELTS / ...
ARRAYELTS    <- VALUE COMMA ARRAYELTS / VALUE
VALUE        <- STRING / FLOAT / INT / BOOL / NULL / HASH / ARRAY
STRING       <- { '"' ( '\\' ([nrtv"] / [0-9]^3) / [^"\\] )* '"' }
INT          <- { [0-9]+ }
FLOAT        <- { [0-9]* '.' [0-9]+ }
BOOL         <- { 'true' / 'false' }
NULL         <- { 'null' }
CBOPEN       <- '{'
CBCLOSE      <- '}'
ABOPEN       <- '['
ABCLOSE      <- ']'
COMMA        <- ','
COLON        <- ':'
END          <- !.
\end{verbatim}
\end{myquote}
\textit{Example of a gpeg grammar meant to validate and capture from
JSON\cite{bib:json} text.}

\subsection{Caveats}

This section highlights a few things to keep in mind when using
gpeg, or PEGs in general.

\subsubsection{Greedy and Blind Behavior}

PEG is, by itself, both greedy and blind.
That is to say, it consumes as much input as possible (greedy), and it does not
look ahead whether a next token is perhaps a better match for the input (blind).
It can however, also implement non-blind behavior, both in a greedy as
well as a non-greedy manner.

Having a greedy, blind parser - standard PEGs - implies that the following
rule will always fail (because the '.*' will consume all input):

\begin{myquote}
\begin{verbatim}
RULE <- .* 'foo'
\end{verbatim}
\end{myquote}

Grammars or patterns like the above, are usually treated in a more
friendly way by, for example, Regex parsers. They implement
a recursive algorithm around 'does the whole pattern match' question,
which backtracks 'from zero to any amount' quantifiers such as the one
used above,
leaving the user with a much more freedom to implement patterns but,
in the end, a much more unpredictable system (for example, what if
'foo' happens twice in the text?)

gpeg is not so understanding. To make it match something while
also doing a lookahead (therefore 'non blind'),
you can implement the following pattern
recursively (non greedy, so catching the first lookahead match):

\begin{myquote}
\begin{verbatim}
S <- E2 / E1 S
\end{verbatim}
\end{myquote}

Or, not recursively:

\begin{myquote}
\begin{verbatim}
S <- ( !E2 E1 )* E2
\end{verbatim}
\end{myquote}

And greedy, so catching only the last lookahead match:

\begin{myquote}
\begin{verbatim}
S <- E1 S / E2
\end{verbatim}
\end{myquote}

\subsubsection{Recursiveness}

Just like with greedy matching, gpeg will simply jump headlong
into your rule's first matcher, and will not try to second-guess your
intentions. So while in many grammar systems, left recursive is
usually considered the best way to describe
repetition, PEG will get stuck in an endless loop if you do the following:

\begin{myquote}
\begin{verbatim}
S <- S SOMEOTHER
\end{verbatim}
\end{myquote}

\subsection{Comments}

A comment starts with two minus signs and ends at the end of the line.

\begin{myquote}
\begin{verbatim}
-- Example of a single line comment.
\end{verbatim}
\end{myquote}

A multiline comment is also avaiable: these start with two minus signs
followed by two angle braces. The multiline comment is closed by two
closing angle braces.

\begin{myquote}
\begin{verbatim}
--[[
     Example of a
     multiline comment.
  ]]
\end{verbatim}
\end{myquote}

\subsection{Rules}

A rule is defined as an \textit{Identifier}, followed by a
left-pointing arrow (composed of a less-than and a minus sign),
followed by a matching expression.

When a gpeg grammar consists of a sequence of rules
(as opposed to a single line expression),
the first rule is used as the starting point for matching inputs.

\begin{myquote}
\begin{verbatim}
RULE1 <- 'a' / RULE2
RULE2 <- 'b' / RULE3
RULE3 <- 'c'
\end{verbatim}
\end{myquote}
\textit{Definition of a set of rules}

\subsubsection{Identifiers}

Identifiers, in gpeg, are defined as a combination of letters,
numbers and the underscore character, not starting with a number,
of between one and 64 characters long.

\begin{myquote}
\begin{verbatim}
IDENTIFIER <- [a-zA-Z_][a-zA-Z0-9_]^-63
\end{verbatim}
\end{myquote}

Identifiers are used to start rule definitions, and as references
to rules in expressions.

\subsubsection{Special Rules}

There is currently one special rule in gpeg grammar: a rule called
'\_\_prefix' will be treated specially. This rule will be, from its
definition onwards, called before the execution of any subsequent rule
definition expression. The aim is to make whitespace and comment filtering
in program language parsing easier. See also the earlier 'JSON' grammar
example.

\subsection{Expressions}

Expressions are lists of terms, optionally separated by the
OR operator (denoted by the forward slash sign).

\begin{myquote}
\begin{verbatim}
ALTERNATIVES <- ALT1 / ALT2 / ALT3
\end{verbatim}
\end{myquote}

A sequence of terms has a higher precendence than the OR operator,
so any other desired order of precedence has to be forced by using
brackets.
In the example below, the expressions evaluate differently:

\begin{myquote}
\begin{verbatim}
RULE1 <- 'a' 'b' / 'c'
RULE2 <- 'a' ( 'b' / 'c' )
\end{verbatim}
\end{myquote}

\subsection{Terms}

A term is defined as a matcher, potentially adorned with prefix
or postfix operators (though not at the same time).
Prefix operators are the NOT and CONFIRM modifiers.
Postfix operators are the quantifiers.

\subsubsection{The NOT Modifier}

The NOT modifier is an exclamation mark.
It does not advance the input position, but succeeds
when matching fails (including reaching the end of input),
and fails when matching succeeds.
Since gpeg is a greedy parser,
this can be used to implement a non-greedy lookahead, for example:
 
\begin{myquote}
\begin{verbatim}
CMULTILINECOMMENT <- '/*' (!'*/' .)* '*/'
\end{verbatim}
\end{myquote}

\subsubsection{The CONFIRM Modifier}

The CONFIRM modifier is an ampersand.
It does not advance the input position, and succeeds
as matching succeeds, and fails when matching fails.
It can be used to process (parts of the) input twice.
It is equivalent to using the NOT modifier twice.

Example where the input is first confirmed to be valid UTF-8,
and only after that, processed again from the beginning, for
well-formedness.
\begin{myquote}
\begin{verbatim}
JSON <- & UTF8 WELLFORMED
\end{verbatim}
\end{myquote}

\subsubsection{Quantifiers}

gpeg denotes quantifiers using a circonflex, followed by either
an absolute number, or a range. Shorthand exists for certain, often-used
ranges. These shorthand ranges are common in Regex as well:

\begin{center}
%\caption{gpeg Quantifiers}
\label{tab:naig_quantifiers}
\begin{longtable}{ll}
\textbf{Quantifier} & \textbf{Semantics} \\
\endhead
* & Zero or more matches \\
+ & One or more matches \\
? & Zero or one matches \\
\^{}\textit{n} & \textit{n} matches \\
\^{}-\textit{n} & Zero up to and including \textit{n} matches \\
\^{}\textit{n}- & \textit{n} matches or more \\
\^{}\textit{n}-\textit{m} & From \textit{n} up to and including \textit{m} matches \\
\end{longtable}
\end{center}
 
Example of the use of quantifiers:
\begin{myquote}
\begin{verbatim}
RULE1 <- 'a'? 'b'^-5
\end{verbatim}
\end{myquote}

\subsection{Matchers}

\subsubsection{The ANY Matcher}

The ANY matcher is denoted (like in Regex) with a single dot.
It matches any character of input and it succeeds - advancing the input
pointer by one - if it can find one at the input pointer. It fails only
at the end of input.

\begin{myquote}
\begin{verbatim}
RUNGREEDILYTOTHEENDOFINPUT <- .*
\end{verbatim}
\end{myquote}

\subsubsection{The SET Matcher}

The ANY matcher, together with the NOT modifier,
also allows you to define end-of-input, like so:

\begin{myquote}
\begin{verbatim}
ENDOFINPUT <- !.
\end{verbatim}
\end{myquote}

\subsubsection{The SET Matcher}

The SET matcher is denoted, like in Regex, as a series of literals
and ranges, enclosed between angled brackets. It can also be negated,
in which case it matches any character not in the set.

\begin{myquote}
\begin{verbatim}
ALPHABETIC     <- [a-zA-Z]
NONALPHABETIC  <- [^a-zA-Z]
\end{verbatim}
\end{myquote}

The denotation of the set elements has a backslash as escape character
(to encode the minus and the closing angled brace).
It also has a form of binary escaping, to encode non-ASCII characters
as part of the set: a backslash followed by three digits of the characters
octal notation.

\subsubsection{The STRING Matcher}

The STRING matcher is denoted as a string literal, between single
quotes. The string may be postfixed with an lower case 'i' to
indicate (alphabetic) case insensitive matching.

\begin{myquote}
\begin{verbatim}
CASEINSENSTIVESTRING <- 'peg'i
\end{verbatim}
\end{myquote}

\subsubsection{The BITMASK Matcher}

The BITMASK matcher allows you to match bits.

\subsubsection{The HEXLITERAL Matcher}

The HEXLITERAL matcher allows you to match single characters by
through their hexadecimal representation.

\subsubsection{The MACRO Matcher}

All macroes are shorthand equivalents of the SET matcher.

\begin{center}
%\caption{gpeg Macroes}
\label{tab:naig_macroes}
\begin{longtable}{lll}
\textbf{Macro} & \textbf{Semantics} & \textbf{Set} \\
\endhead
\%s & Whitespace & [ \textbackslash n\textbackslash r\textbackslash t\textbackslash v] \\
\%w & Alphabetic & [a-zA-Z] \\
\%a & Alphanumeric & [a-zA-Z0-9] \\
\%n & Numeric & [0-9] \\
\%p & Printable & 0x20 - 0x7f \\
\end{longtable}
\end{center}

\subsubsection{The GROUP Matcher}

\subsubsection{The CAPTURE Matcher}

\subsubsection{The VARCAPTURE Matcher}

\subsubsection{The VARREFERENCE Matcher}

\subsubsection{The REFERENCE Matcher}

\subsubsection{The LIMITEDINPUT Matcher}

\subsubsection{The END Matcher}

