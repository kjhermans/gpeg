\section{File Formats}

This section describes all the user available memory and file formats
associated with gpeg.

\subsection{Grammar}

The gpeg grammar text format is descibed in [\ref{sec:grammar}].
Various pre-compilers can potentially produce these format however,
these are out of scope of this documentation. The grammar format is
consumed by the gpeg assembler, gpegc. For the use of gpegc, see
[\ref{exe:gpegc}].

\subsection{Assembly}

The gpeg assembly text format is descibed in [\ref{sec:assembly}].
This format is produced by the gpeg compiler, gpegc,
and is consumed by the gpeg assembler, gpega.
For the use of gpega, see [\ref{exe:gpega}].

\subsection{Maps}

When gpeg tooling outputs maps (and there are two: slotmaps and labelmaps)
it does so in a single, simple textual format: rows of an identifier followed
by a space followed by a number (followed by a new line).

For example:

\begin{myquote}
\begin{verbatim}

//.. TODO

\end{verbatim}
\end{myquote}

\subsection{Slotmap.h}

The gpeg compiler assigns a slot number to each capture it encounters.
It then assigns an internal unique name to those capture numbers, or slots.
Finally, the compiler can be requested, through the command line, to
'rain down' these named mappings to slot number in a C-header file, so that
API users can use easy-to-remember \#defines to address their captures.

\subsection{Disassembly}

The gpeg disassembler, gpegd, will take bytecode and reproduce the
assembly. This assembly will be different from the original assembly
in that:

\begin{itemize}
\item It will not contain textually understandable labels.
\item It will prefix every instruction with a numeric label, which is
      equivalent to the bytecode offset of the instruction's opcode.
\item These offsets will then also be used in jumps.
\end{itemize}

\subsection{Engine Debug Logs}

Running the gpeg engine, gpege, in debug mode will yield, on standard error,
a log, which will contain, per instruction executed, a line representing
the engine's internal state, most notably its stack, like so:

\begin{myquote}
\begin{verbatim}
CHAR bc 2316 in 482 0403020100______ st (014 prec.) ALT:1484
CLL:1476 C LL:1820 ALT:1944 CLL:1928 ALT:1652 CLL:1644 CLL:2440
\end{verbatim}
\end{myquote}

These lines are built up as follows:

\begin{itemize}
\item The instruction being executed (in this case, 'char').
\item The bytecode offset (in this case, 2316 decimal).
\item The input offset (in this case, 482 decimal).
\item A subset of the input, from the input offset (in this in hexadecimal).
\item An overview of the stack (in this case, with 14 preceding items left
      out), followed by the last stack items, either 'call' items
      (with their return addresses), or 'alternative' items
      (with their jump-to addresses in case they catch a FAIL condition).
\end{itemize}

This section describes the formats used by the gpeg tooling, other than
the ones described in the sections on grammar, assembly and bytecode.

\subsection{Slotmap Format}

The slotmap file exists to help developers by easing access to
captures in complex grammars.

The gpeg grammar compiler can be instructed to emit a slotmap file.
This is a file which maps a unique name to a capture region's unique index.
This is provided so that, instead of using the index of capture region
(which requires hand counting them in your grammar file, which can be
tedious and error-prone, and something that would not survive
a grammar reshuffle, or the introduction of a capture region before
the one you're interested in), you can use a naming scheme for your
capture regions.

Names in the slotmap file are bound semantically to the capture region:
they are made up of the name of the rule, optionally followed by the
index of the capture region within the rule's definition.

For example, the following rule with capture region definition:

\begin{myquote}
\begin{verbatim}
RULE <- { IDENT } OPTARGS LEFTARROW EXPRESSION
\end{verbatim}
\end{myquote}

will result in slotmap identifier 'RULE\_0'.

\subsection{Labelmap Format}

The labelmap format is optionally emitted by the assembler, and exists to:

\begin{itemize}
\item Allow for better debugging, because offsets in bytecode can be reduced
      to more intuitively named sections (rule names always make it to
      the labelmap unchanged).
\item Allow for calling the bytecode at symbols directly, which in turn
      allows you to use a bytecode blob more as a database of functions.
\end{itemize}

\subsection{Engine Output Format}

The gpeg bytecode execution engine, gpege, produces, on success, output for
digital processing, in the form of a binary table, which is structured
as follows:

\begin{itemize}

\item Each record is four 32 bit integers, in network order.

\item The first record contains the end code of the matching process.
This is the same code as was given as a parameter to the 'end'
instruction that resulted in the execution finishing. This is the
first field, by default it is zero.
The second field contains the amount of subsequent records.

\item Subsequent records have as fields, either:

\begin{itemize}
\item Type, slot, start, length (when of 'capture' type), or:
\item Type, slot, start, char (when of 'replace' type).
\end{itemize}

\end{itemize}

Bear in mind the following:

\begin{itemize}
\item 'Capture' type is denoted as 1, 'Replace' as 3.
\item Offsets and lengths of captures refer to the input buffer.
\item Offsets and lengths of replacements refer to the bytecode.
\end{itemize}

\begin{myquote}
\begin{verbatim}
00 00 00 00 00 00 03 64 00 00 00 00 00 00 00 00  .......d........
00 00 00 01 00 00 00 00 00 00 00 1c 00 00 00 23  ...............#
00 00 00 01 00 00 00 03 00 00 00 32 00 00 00 59  ...........2...Y
00 00 00 01 00 00 00 04 00 00 00 32 00 00 00 55  ...........2...U
00 00 00 01 00 00 00 14 00 00 00 32 00 00 00 55  ...........2...U
00 00 00 01 00 00 00 02 00 00 00 34 00 00 00 3f  ...........4...?
00 00 00 01 00 00 00 04 00 00 00 34 00 00 00 3f  ...........4...?
00 00 00 01 00 00 00 17 00 00 00 34 00 00 00 3e  ...........4...>
00 00 00 01 00 00 00 06 00 00 00 3e 00 00 00 3f  ...........>...?
00 00 00 01 00 00 00 03 00 00 00 42 00 00 00 53  ...........B...S
00 00 00 01 00 00 00 04 00 00 00 42 00 00 00 53  ...........B...S
00 00 00 01 00 00 00 17 00 00 00 42 00 00 00 53  ...........B...S
\end{verbatim}
\end{myquote}
\textit{Example of the head of engine output with a zero end code
and 868 matches (all captures), hex dumped}
